
% Default to the notebook output style

    


% Inherit from the specified cell style.




    
\documentclass[11pt]{article}

    
    
    \usepackage[T1]{fontenc}
    % Nicer default font (+ math font) than Computer Modern for most use cases
    \usepackage{mathpazo}

    % Basic figure setup, for now with no caption control since it's done
    % automatically by Pandoc (which extracts ![](path) syntax from Markdown).
    \usepackage{graphicx}
    % We will generate all images so they have a width \maxwidth. This means
    % that they will get their normal width if they fit onto the page, but
    % are scaled down if they would overflow the margins.
    \makeatletter
    \def\maxwidth{\ifdim\Gin@nat@width>\linewidth\linewidth
    \else\Gin@nat@width\fi}
    \makeatother
    \let\Oldincludegraphics\includegraphics
    % Set max figure width to be 80% of text width, for now hardcoded.
    \renewcommand{\includegraphics}[1]{\Oldincludegraphics[width=.8\maxwidth]{#1}}
    % Ensure that by default, figures have no caption (until we provide a
    % proper Figure object with a Caption API and a way to capture that
    % in the conversion process - todo).
    \usepackage{caption}
    \DeclareCaptionLabelFormat{nolabel}{}
    \captionsetup{labelformat=nolabel}

    \usepackage{adjustbox} % Used to constrain images to a maximum size 
    \usepackage{xcolor} % Allow colors to be defined
    \usepackage{enumerate} % Needed for markdown enumerations to work
    \usepackage{geometry} % Used to adjust the document margins
    \usepackage{amsmath} % Equations
    \usepackage{amssymb} % Equations
    \usepackage{textcomp} % defines textquotesingle
    % Hack from http://tex.stackexchange.com/a/47451/13684:
    \AtBeginDocument{%
        \def\PYZsq{\textquotesingle}% Upright quotes in Pygmentized code
    }
    \usepackage{upquote} % Upright quotes for verbatim code
    \usepackage{eurosym} % defines \euro
    \usepackage[mathletters]{ucs} % Extended unicode (utf-8) support
    \usepackage[utf8x]{inputenc} % Allow utf-8 characters in the tex document
    \usepackage{fancyvrb} % verbatim replacement that allows latex
    \usepackage{grffile} % extends the file name processing of package graphics 
                         % to support a larger range 
    % The hyperref package gives us a pdf with properly built
    % internal navigation ('pdf bookmarks' for the table of contents,
    % internal cross-reference links, web links for URLs, etc.)
    \usepackage{hyperref}
    \usepackage{longtable} % longtable support required by pandoc >1.10
    \usepackage{booktabs}  % table support for pandoc > 1.12.2
    \usepackage[inline]{enumitem} % IRkernel/repr support (it uses the enumerate* environment)
    \usepackage[normalem]{ulem} % ulem is needed to support strikethroughs (\sout)
                                % normalem makes italics be italics, not underlines
    

    
    
    % Colors for the hyperref package
    \definecolor{urlcolor}{rgb}{0,.145,.698}
    \definecolor{linkcolor}{rgb}{.71,0.21,0.01}
    \definecolor{citecolor}{rgb}{.12,.54,.11}

    % ANSI colors
    \definecolor{ansi-black}{HTML}{3E424D}
    \definecolor{ansi-black-intense}{HTML}{282C36}
    \definecolor{ansi-red}{HTML}{E75C58}
    \definecolor{ansi-red-intense}{HTML}{B22B31}
    \definecolor{ansi-green}{HTML}{00A250}
    \definecolor{ansi-green-intense}{HTML}{007427}
    \definecolor{ansi-yellow}{HTML}{DDB62B}
    \definecolor{ansi-yellow-intense}{HTML}{B27D12}
    \definecolor{ansi-blue}{HTML}{208FFB}
    \definecolor{ansi-blue-intense}{HTML}{0065CA}
    \definecolor{ansi-magenta}{HTML}{D160C4}
    \definecolor{ansi-magenta-intense}{HTML}{A03196}
    \definecolor{ansi-cyan}{HTML}{60C6C8}
    \definecolor{ansi-cyan-intense}{HTML}{258F8F}
    \definecolor{ansi-white}{HTML}{C5C1B4}
    \definecolor{ansi-white-intense}{HTML}{A1A6B2}

    % commands and environments needed by pandoc snippets
    % extracted from the output of `pandoc -s`
    \providecommand{\tightlist}{%
      \setlength{\itemsep}{0pt}\setlength{\parskip}{0pt}}
    \DefineVerbatimEnvironment{Highlighting}{Verbatim}{commandchars=\\\{\}}
    % Add ',fontsize=\small' for more characters per line
    \newenvironment{Shaded}{}{}
    \newcommand{\KeywordTok}[1]{\textcolor[rgb]{0.00,0.44,0.13}{\textbf{{#1}}}}
    \newcommand{\DataTypeTok}[1]{\textcolor[rgb]{0.56,0.13,0.00}{{#1}}}
    \newcommand{\DecValTok}[1]{\textcolor[rgb]{0.25,0.63,0.44}{{#1}}}
    \newcommand{\BaseNTok}[1]{\textcolor[rgb]{0.25,0.63,0.44}{{#1}}}
    \newcommand{\FloatTok}[1]{\textcolor[rgb]{0.25,0.63,0.44}{{#1}}}
    \newcommand{\CharTok}[1]{\textcolor[rgb]{0.25,0.44,0.63}{{#1}}}
    \newcommand{\StringTok}[1]{\textcolor[rgb]{0.25,0.44,0.63}{{#1}}}
    \newcommand{\CommentTok}[1]{\textcolor[rgb]{0.38,0.63,0.69}{\textit{{#1}}}}
    \newcommand{\OtherTok}[1]{\textcolor[rgb]{0.00,0.44,0.13}{{#1}}}
    \newcommand{\AlertTok}[1]{\textcolor[rgb]{1.00,0.00,0.00}{\textbf{{#1}}}}
    \newcommand{\FunctionTok}[1]{\textcolor[rgb]{0.02,0.16,0.49}{{#1}}}
    \newcommand{\RegionMarkerTok}[1]{{#1}}
    \newcommand{\ErrorTok}[1]{\textcolor[rgb]{1.00,0.00,0.00}{\textbf{{#1}}}}
    \newcommand{\NormalTok}[1]{{#1}}
    
    % Additional commands for more recent versions of Pandoc
    \newcommand{\ConstantTok}[1]{\textcolor[rgb]{0.53,0.00,0.00}{{#1}}}
    \newcommand{\SpecialCharTok}[1]{\textcolor[rgb]{0.25,0.44,0.63}{{#1}}}
    \newcommand{\VerbatimStringTok}[1]{\textcolor[rgb]{0.25,0.44,0.63}{{#1}}}
    \newcommand{\SpecialStringTok}[1]{\textcolor[rgb]{0.73,0.40,0.53}{{#1}}}
    \newcommand{\ImportTok}[1]{{#1}}
    \newcommand{\DocumentationTok}[1]{\textcolor[rgb]{0.73,0.13,0.13}{\textit{{#1}}}}
    \newcommand{\AnnotationTok}[1]{\textcolor[rgb]{0.38,0.63,0.69}{\textbf{\textit{{#1}}}}}
    \newcommand{\CommentVarTok}[1]{\textcolor[rgb]{0.38,0.63,0.69}{\textbf{\textit{{#1}}}}}
    \newcommand{\VariableTok}[1]{\textcolor[rgb]{0.10,0.09,0.49}{{#1}}}
    \newcommand{\ControlFlowTok}[1]{\textcolor[rgb]{0.00,0.44,0.13}{\textbf{{#1}}}}
    \newcommand{\OperatorTok}[1]{\textcolor[rgb]{0.40,0.40,0.40}{{#1}}}
    \newcommand{\BuiltInTok}[1]{{#1}}
    \newcommand{\ExtensionTok}[1]{{#1}}
    \newcommand{\PreprocessorTok}[1]{\textcolor[rgb]{0.74,0.48,0.00}{{#1}}}
    \newcommand{\AttributeTok}[1]{\textcolor[rgb]{0.49,0.56,0.16}{{#1}}}
    \newcommand{\InformationTok}[1]{\textcolor[rgb]{0.38,0.63,0.69}{\textbf{\textit{{#1}}}}}
    \newcommand{\WarningTok}[1]{\textcolor[rgb]{0.38,0.63,0.69}{\textbf{\textit{{#1}}}}}
    
    
    % Define a nice break command that doesn't care if a line doesn't already
    % exist.
    \def\br{\hspace*{\fill} \\* }
    % Math Jax compatability definitions
    \def\gt{>}
    \def\lt{<}
    % Document parameters
    \title{ProblemSolving}
    
    
    

    % Pygments definitions
    
\makeatletter
\def\PY@reset{\let\PY@it=\relax \let\PY@bf=\relax%
    \let\PY@ul=\relax \let\PY@tc=\relax%
    \let\PY@bc=\relax \let\PY@ff=\relax}
\def\PY@tok#1{\csname PY@tok@#1\endcsname}
\def\PY@toks#1+{\ifx\relax#1\empty\else%
    \PY@tok{#1}\expandafter\PY@toks\fi}
\def\PY@do#1{\PY@bc{\PY@tc{\PY@ul{%
    \PY@it{\PY@bf{\PY@ff{#1}}}}}}}
\def\PY#1#2{\PY@reset\PY@toks#1+\relax+\PY@do{#2}}

\expandafter\def\csname PY@tok@w\endcsname{\def\PY@tc##1{\textcolor[rgb]{0.73,0.73,0.73}{##1}}}
\expandafter\def\csname PY@tok@c\endcsname{\let\PY@it=\textit\def\PY@tc##1{\textcolor[rgb]{0.25,0.50,0.50}{##1}}}
\expandafter\def\csname PY@tok@cp\endcsname{\def\PY@tc##1{\textcolor[rgb]{0.74,0.48,0.00}{##1}}}
\expandafter\def\csname PY@tok@k\endcsname{\let\PY@bf=\textbf\def\PY@tc##1{\textcolor[rgb]{0.00,0.50,0.00}{##1}}}
\expandafter\def\csname PY@tok@kp\endcsname{\def\PY@tc##1{\textcolor[rgb]{0.00,0.50,0.00}{##1}}}
\expandafter\def\csname PY@tok@kt\endcsname{\def\PY@tc##1{\textcolor[rgb]{0.69,0.00,0.25}{##1}}}
\expandafter\def\csname PY@tok@o\endcsname{\def\PY@tc##1{\textcolor[rgb]{0.40,0.40,0.40}{##1}}}
\expandafter\def\csname PY@tok@ow\endcsname{\let\PY@bf=\textbf\def\PY@tc##1{\textcolor[rgb]{0.67,0.13,1.00}{##1}}}
\expandafter\def\csname PY@tok@nb\endcsname{\def\PY@tc##1{\textcolor[rgb]{0.00,0.50,0.00}{##1}}}
\expandafter\def\csname PY@tok@nf\endcsname{\def\PY@tc##1{\textcolor[rgb]{0.00,0.00,1.00}{##1}}}
\expandafter\def\csname PY@tok@nc\endcsname{\let\PY@bf=\textbf\def\PY@tc##1{\textcolor[rgb]{0.00,0.00,1.00}{##1}}}
\expandafter\def\csname PY@tok@nn\endcsname{\let\PY@bf=\textbf\def\PY@tc##1{\textcolor[rgb]{0.00,0.00,1.00}{##1}}}
\expandafter\def\csname PY@tok@ne\endcsname{\let\PY@bf=\textbf\def\PY@tc##1{\textcolor[rgb]{0.82,0.25,0.23}{##1}}}
\expandafter\def\csname PY@tok@nv\endcsname{\def\PY@tc##1{\textcolor[rgb]{0.10,0.09,0.49}{##1}}}
\expandafter\def\csname PY@tok@no\endcsname{\def\PY@tc##1{\textcolor[rgb]{0.53,0.00,0.00}{##1}}}
\expandafter\def\csname PY@tok@nl\endcsname{\def\PY@tc##1{\textcolor[rgb]{0.63,0.63,0.00}{##1}}}
\expandafter\def\csname PY@tok@ni\endcsname{\let\PY@bf=\textbf\def\PY@tc##1{\textcolor[rgb]{0.60,0.60,0.60}{##1}}}
\expandafter\def\csname PY@tok@na\endcsname{\def\PY@tc##1{\textcolor[rgb]{0.49,0.56,0.16}{##1}}}
\expandafter\def\csname PY@tok@nt\endcsname{\let\PY@bf=\textbf\def\PY@tc##1{\textcolor[rgb]{0.00,0.50,0.00}{##1}}}
\expandafter\def\csname PY@tok@nd\endcsname{\def\PY@tc##1{\textcolor[rgb]{0.67,0.13,1.00}{##1}}}
\expandafter\def\csname PY@tok@s\endcsname{\def\PY@tc##1{\textcolor[rgb]{0.73,0.13,0.13}{##1}}}
\expandafter\def\csname PY@tok@sd\endcsname{\let\PY@it=\textit\def\PY@tc##1{\textcolor[rgb]{0.73,0.13,0.13}{##1}}}
\expandafter\def\csname PY@tok@si\endcsname{\let\PY@bf=\textbf\def\PY@tc##1{\textcolor[rgb]{0.73,0.40,0.53}{##1}}}
\expandafter\def\csname PY@tok@se\endcsname{\let\PY@bf=\textbf\def\PY@tc##1{\textcolor[rgb]{0.73,0.40,0.13}{##1}}}
\expandafter\def\csname PY@tok@sr\endcsname{\def\PY@tc##1{\textcolor[rgb]{0.73,0.40,0.53}{##1}}}
\expandafter\def\csname PY@tok@ss\endcsname{\def\PY@tc##1{\textcolor[rgb]{0.10,0.09,0.49}{##1}}}
\expandafter\def\csname PY@tok@sx\endcsname{\def\PY@tc##1{\textcolor[rgb]{0.00,0.50,0.00}{##1}}}
\expandafter\def\csname PY@tok@m\endcsname{\def\PY@tc##1{\textcolor[rgb]{0.40,0.40,0.40}{##1}}}
\expandafter\def\csname PY@tok@gh\endcsname{\let\PY@bf=\textbf\def\PY@tc##1{\textcolor[rgb]{0.00,0.00,0.50}{##1}}}
\expandafter\def\csname PY@tok@gu\endcsname{\let\PY@bf=\textbf\def\PY@tc##1{\textcolor[rgb]{0.50,0.00,0.50}{##1}}}
\expandafter\def\csname PY@tok@gd\endcsname{\def\PY@tc##1{\textcolor[rgb]{0.63,0.00,0.00}{##1}}}
\expandafter\def\csname PY@tok@gi\endcsname{\def\PY@tc##1{\textcolor[rgb]{0.00,0.63,0.00}{##1}}}
\expandafter\def\csname PY@tok@gr\endcsname{\def\PY@tc##1{\textcolor[rgb]{1.00,0.00,0.00}{##1}}}
\expandafter\def\csname PY@tok@ge\endcsname{\let\PY@it=\textit}
\expandafter\def\csname PY@tok@gs\endcsname{\let\PY@bf=\textbf}
\expandafter\def\csname PY@tok@gp\endcsname{\let\PY@bf=\textbf\def\PY@tc##1{\textcolor[rgb]{0.00,0.00,0.50}{##1}}}
\expandafter\def\csname PY@tok@go\endcsname{\def\PY@tc##1{\textcolor[rgb]{0.53,0.53,0.53}{##1}}}
\expandafter\def\csname PY@tok@gt\endcsname{\def\PY@tc##1{\textcolor[rgb]{0.00,0.27,0.87}{##1}}}
\expandafter\def\csname PY@tok@err\endcsname{\def\PY@bc##1{\setlength{\fboxsep}{0pt}\fcolorbox[rgb]{1.00,0.00,0.00}{1,1,1}{\strut ##1}}}
\expandafter\def\csname PY@tok@kc\endcsname{\let\PY@bf=\textbf\def\PY@tc##1{\textcolor[rgb]{0.00,0.50,0.00}{##1}}}
\expandafter\def\csname PY@tok@kd\endcsname{\let\PY@bf=\textbf\def\PY@tc##1{\textcolor[rgb]{0.00,0.50,0.00}{##1}}}
\expandafter\def\csname PY@tok@kn\endcsname{\let\PY@bf=\textbf\def\PY@tc##1{\textcolor[rgb]{0.00,0.50,0.00}{##1}}}
\expandafter\def\csname PY@tok@kr\endcsname{\let\PY@bf=\textbf\def\PY@tc##1{\textcolor[rgb]{0.00,0.50,0.00}{##1}}}
\expandafter\def\csname PY@tok@bp\endcsname{\def\PY@tc##1{\textcolor[rgb]{0.00,0.50,0.00}{##1}}}
\expandafter\def\csname PY@tok@fm\endcsname{\def\PY@tc##1{\textcolor[rgb]{0.00,0.00,1.00}{##1}}}
\expandafter\def\csname PY@tok@vc\endcsname{\def\PY@tc##1{\textcolor[rgb]{0.10,0.09,0.49}{##1}}}
\expandafter\def\csname PY@tok@vg\endcsname{\def\PY@tc##1{\textcolor[rgb]{0.10,0.09,0.49}{##1}}}
\expandafter\def\csname PY@tok@vi\endcsname{\def\PY@tc##1{\textcolor[rgb]{0.10,0.09,0.49}{##1}}}
\expandafter\def\csname PY@tok@vm\endcsname{\def\PY@tc##1{\textcolor[rgb]{0.10,0.09,0.49}{##1}}}
\expandafter\def\csname PY@tok@sa\endcsname{\def\PY@tc##1{\textcolor[rgb]{0.73,0.13,0.13}{##1}}}
\expandafter\def\csname PY@tok@sb\endcsname{\def\PY@tc##1{\textcolor[rgb]{0.73,0.13,0.13}{##1}}}
\expandafter\def\csname PY@tok@sc\endcsname{\def\PY@tc##1{\textcolor[rgb]{0.73,0.13,0.13}{##1}}}
\expandafter\def\csname PY@tok@dl\endcsname{\def\PY@tc##1{\textcolor[rgb]{0.73,0.13,0.13}{##1}}}
\expandafter\def\csname PY@tok@s2\endcsname{\def\PY@tc##1{\textcolor[rgb]{0.73,0.13,0.13}{##1}}}
\expandafter\def\csname PY@tok@sh\endcsname{\def\PY@tc##1{\textcolor[rgb]{0.73,0.13,0.13}{##1}}}
\expandafter\def\csname PY@tok@s1\endcsname{\def\PY@tc##1{\textcolor[rgb]{0.73,0.13,0.13}{##1}}}
\expandafter\def\csname PY@tok@mb\endcsname{\def\PY@tc##1{\textcolor[rgb]{0.40,0.40,0.40}{##1}}}
\expandafter\def\csname PY@tok@mf\endcsname{\def\PY@tc##1{\textcolor[rgb]{0.40,0.40,0.40}{##1}}}
\expandafter\def\csname PY@tok@mh\endcsname{\def\PY@tc##1{\textcolor[rgb]{0.40,0.40,0.40}{##1}}}
\expandafter\def\csname PY@tok@mi\endcsname{\def\PY@tc##1{\textcolor[rgb]{0.40,0.40,0.40}{##1}}}
\expandafter\def\csname PY@tok@il\endcsname{\def\PY@tc##1{\textcolor[rgb]{0.40,0.40,0.40}{##1}}}
\expandafter\def\csname PY@tok@mo\endcsname{\def\PY@tc##1{\textcolor[rgb]{0.40,0.40,0.40}{##1}}}
\expandafter\def\csname PY@tok@ch\endcsname{\let\PY@it=\textit\def\PY@tc##1{\textcolor[rgb]{0.25,0.50,0.50}{##1}}}
\expandafter\def\csname PY@tok@cm\endcsname{\let\PY@it=\textit\def\PY@tc##1{\textcolor[rgb]{0.25,0.50,0.50}{##1}}}
\expandafter\def\csname PY@tok@cpf\endcsname{\let\PY@it=\textit\def\PY@tc##1{\textcolor[rgb]{0.25,0.50,0.50}{##1}}}
\expandafter\def\csname PY@tok@c1\endcsname{\let\PY@it=\textit\def\PY@tc##1{\textcolor[rgb]{0.25,0.50,0.50}{##1}}}
\expandafter\def\csname PY@tok@cs\endcsname{\let\PY@it=\textit\def\PY@tc##1{\textcolor[rgb]{0.25,0.50,0.50}{##1}}}

\def\PYZbs{\char`\\}
\def\PYZus{\char`\_}
\def\PYZob{\char`\{}
\def\PYZcb{\char`\}}
\def\PYZca{\char`\^}
\def\PYZam{\char`\&}
\def\PYZlt{\char`\<}
\def\PYZgt{\char`\>}
\def\PYZsh{\char`\#}
\def\PYZpc{\char`\%}
\def\PYZdl{\char`\$}
\def\PYZhy{\char`\-}
\def\PYZsq{\char`\'}
\def\PYZdq{\char`\"}
\def\PYZti{\char`\~}
% for compatibility with earlier versions
\def\PYZat{@}
\def\PYZlb{[}
\def\PYZrb{]}
\makeatother


    % Exact colors from NB
    \definecolor{incolor}{rgb}{0.0, 0.0, 0.5}
    \definecolor{outcolor}{rgb}{0.545, 0.0, 0.0}



    
    % Prevent overflowing lines due to hard-to-break entities
    \sloppy 
    % Setup hyperref package
    \hypersetup{
      breaklinks=true,  % so long urls are correctly broken across lines
      colorlinks=true,
      urlcolor=urlcolor,
      linkcolor=linkcolor,
      citecolor=citecolor,
      }
    % Slightly bigger margins than the latex defaults
    
    \geometry{verbose,tmargin=1in,bmargin=1in,lmargin=1in,rmargin=1in}
    
    

    \begin{document}
    
    
    \maketitle
    
    

    
    \textbf{Problem Statement (A valid password)}: In an application a valid
password must be a combination of digits, uppercase and lowercase
letters and only four symbols * @ ! ? . The length of the password must
not be less than 8 characters. In case the password is not valid, the
user can try several times until it is accepted.

    We are going to use this case study to practice with the basics of
computational problem solving techniques which any junior computer
science student needs to practice during his/her study.

Given any problem a student is expected to iteratively and
systematically take the following steps. - Formulating the problem
statement. - Proposing a candidate model for a solution. - Designing an
algorithm. - Implementing the solution. - Aanalysing the results.

\textbf{Formulating the problem statement}: The user is expected to
enter a string which represents the password. The string which consists
a set of characters must meet some constraints. Each constraint can be
written as a propositional statement. The first constraint checks the
length of the input which is very straitforward to check. The other
constratins simly specifying that the given input must have at least one
memeber from some sets of characters.

\textbf{Proposing a candidate model for a solution}: How can we check
membership of the characters from the given input within some sets?
Let's define our sets: - upp: Uppercase ascii characters. - low:
Lowercase ascii characters. - symb: \{*,@,!,?\} - dgt: The set of
digits. - psw: The set of characters from given input.

Let's formulate constraints. - At least one symbol (from symb):
\[| psw \cap symb | > 0\] - Combination of digits, uppercase and
lowercase letters:
\[| psw \cap upp | > 0 \wedge | psw \cap low | > 0 \wedge | psw \cap dgt | > 0\]

The follwoing picture shows a Venn diagram of our solution:

\textbf{Designing an algorithm}: So far we could manage to build the
core of our solution. What are the steps needed to tranlate our solution
to a program? Let's try. This is our first try: - Assume a given input.
- Build required sets. - Construct given constraints. - Check if
conditions are met: Yes, confirm validity; No, repeat again.

The flow of the execution will look like:

\textbf{Implementing the solution}: First, implement a simple version of
the solution. As soon as we start with coding we will see that there is
a need to initialize our sets. Writing all the characters uppercase,
lowrcase, digits is not fun. There should be a simpler way. Check here
https://docs.python.org/2/library/string.html . Python gives us all the
sets. Try them:

    \begin{Verbatim}[commandchars=\\\{\}]
{\color{incolor}In [{\color{incolor}7}]:} \PY{k+kn}{import} \PY{n+nn}{string}
        
        \PY{n+nb}{print}\PY{p}{(}\PY{n}{string}\PY{o}{.}\PY{n}{ascii\PYZus{}uppercase}\PY{p}{)}
        \PY{n+nb}{print}\PY{p}{(}\PY{n}{string}\PY{o}{.}\PY{n}{ascii\PYZus{}lowercase}\PY{p}{)}
        \PY{n+nb}{print}\PY{p}{(}\PY{n}{string}\PY{o}{.}\PY{n}{digits}\PY{p}{)}
\end{Verbatim}


    \begin{Verbatim}[commandchars=\\\{\}]
ABCDEFGHIJKLMNOPQRSTUVWXYZ
abcdefghijklmnopqrstuvwxyz
0123456789

    \end{Verbatim}

    Now let's implement the basic and simple version of our program.

    \begin{Verbatim}[commandchars=\\\{\}]
{\color{incolor}In [{\color{incolor}2}]:} \PY{k+kn}{import} \PY{n+nn}{string}
        
        \PY{n}{upp} \PY{o}{=} \PY{n+nb}{set}\PY{p}{(}\PY{n}{string}\PY{o}{.}\PY{n}{ascii\PYZus{}uppercase}\PY{p}{)}
        \PY{n}{low} \PY{o}{=} \PY{n+nb}{set}\PY{p}{(}\PY{n}{string}\PY{o}{.}\PY{n}{ascii\PYZus{}lowercase}\PY{p}{)}
        \PY{n}{dgt} \PY{o}{=} \PY{n+nb}{set}\PY{p}{(}\PY{n}{string}\PY{o}{.}\PY{n}{digits}\PY{p}{)}
        \PY{n}{sym} \PY{o}{=} \PY{p}{\PYZob{}}\PY{l+s+s1}{\PYZsq{}}\PY{l+s+s1}{@}\PY{l+s+s1}{\PYZsq{}}\PY{p}{,} \PY{l+s+s1}{\PYZsq{}}\PY{l+s+s1}{?}\PY{l+s+s1}{\PYZsq{}}\PY{p}{,} \PY{l+s+s1}{\PYZsq{}}\PY{l+s+s1}{!}\PY{l+s+s1}{\PYZsq{}}\PY{p}{,} \PY{l+s+s1}{\PYZsq{}}\PY{l+s+s1}{*}\PY{l+s+s1}{\PYZsq{}}\PY{p}{\PYZcb{}}
        
        \PY{n}{cond} \PY{o}{=} \PY{k+kc}{False}
        \PY{k}{while}\PY{p}{(}\PY{n}{cond} \PY{o}{==} \PY{k+kc}{False}\PY{p}{)}\PY{p}{:}
          \PY{n}{psw} \PY{o}{=} \PY{n+nb}{input}\PY{p}{(}\PY{l+s+s1}{\PYZsq{}}\PY{l+s+s1}{Define a password:}\PY{l+s+s1}{\PYZsq{}}\PY{p}{)}
          \PY{n}{psw\PYZus{}set} \PY{o}{=} \PY{n+nb}{set}\PY{p}{(}\PY{n}{psw}\PY{p}{)}
          
          \PY{n}{conds0} \PY{o}{=} \PY{n+nb}{len}\PY{p}{(}\PY{n}{psw}\PY{p}{)} \PY{o}{\PYZgt{}}\PY{o}{=} \PY{l+m+mi}{8} 
          \PY{n}{conds1} \PY{o}{=} \PY{n+nb}{len}\PY{p}{(}\PY{n}{psw\PYZus{}set}\PY{o}{.}\PY{n}{intersection}\PY{p}{(}\PY{n}{dgt}\PY{p}{)}\PY{p}{)} \PY{o}{\PYZgt{}} \PY{l+m+mi}{0}
          \PY{n}{conds2} \PY{o}{=} \PY{n+nb}{len}\PY{p}{(}\PY{n}{psw\PYZus{}set}\PY{o}{.}\PY{n}{intersection}\PY{p}{(}\PY{n}{sym}\PY{p}{)}\PY{p}{)} \PY{o}{\PYZgt{}} \PY{l+m+mi}{0}
          \PY{n}{conds3} \PY{o}{=} \PY{n+nb}{len}\PY{p}{(}\PY{n}{psw\PYZus{}set}\PY{o}{.}\PY{n}{intersection}\PY{p}{(}\PY{n}{upp}\PY{p}{)}\PY{p}{)} \PY{o}{\PYZgt{}} \PY{l+m+mi}{0}
          \PY{n}{conds4} \PY{o}{=} \PY{n+nb}{len}\PY{p}{(}\PY{n}{psw\PYZus{}set}\PY{o}{.}\PY{n}{intersection}\PY{p}{(}\PY{n}{low}\PY{p}{)}\PY{p}{)} \PY{o}{\PYZgt{}} \PY{l+m+mi}{0}
        
          \PY{n}{cond} \PY{o}{=} \PY{n}{conds0} \PY{o+ow}{and} \PY{n}{conds1} \PY{o+ow}{and} \PY{n}{conds2} \PY{o+ow}{and} \PY{n}{conds3} \PY{o+ow}{and} \PY{n}{conds4}
        
          \PY{k}{if} \PY{o+ow}{not}\PY{p}{(}\PY{n}{cond}\PY{p}{)}\PY{p}{:}
            \PY{n+nb}{print}\PY{p}{(}\PY{l+s+s1}{\PYZsq{}}\PY{l+s+s1}{invalid password: enter again}\PY{l+s+s1}{\PYZsq{}}\PY{p}{)}
        
        \PY{n+nb}{print}\PY{p}{(}\PY{l+s+s1}{\PYZsq{}}\PY{l+s+s1}{Password is Valid.}\PY{l+s+s1}{\PYZsq{}}\PY{p}{)}
\end{Verbatim}


    \begin{Verbatim}[commandchars=\\\{\}]
Define a password:Abx12*
invalid password: enter again
Define a password:Ab12*weG!
Password is Valid.

    \end{Verbatim}

    We simply built the main core part of our program. Let's spend some time
to improve our code: - Can we use a better structure for conditions? -
Is there a simple way to report a proper message in case of invalid
pasword? Why given password is not valid? Below, I have change the code
using lists and tuples to implement our extensions:

    \begin{Verbatim}[commandchars=\\\{\}]
{\color{incolor}In [{\color{incolor}5}]:} \PY{k+kn}{import} \PY{n+nn}{string}
        
        \PY{n}{upp} \PY{o}{=} \PY{n+nb}{set}\PY{p}{(}\PY{n}{string}\PY{o}{.}\PY{n}{ascii\PYZus{}uppercase}\PY{p}{)}
        \PY{n}{low} \PY{o}{=} \PY{n+nb}{set}\PY{p}{(}\PY{n}{string}\PY{o}{.}\PY{n}{ascii\PYZus{}lowercase}\PY{p}{)}
        \PY{n}{dgt} \PY{o}{=} \PY{n+nb}{set}\PY{p}{(}\PY{n}{string}\PY{o}{.}\PY{n}{digits}\PY{p}{)}
        \PY{n}{sym} \PY{o}{=} \PY{p}{\PYZob{}}\PY{l+s+s1}{\PYZsq{}}\PY{l+s+s1}{@}\PY{l+s+s1}{\PYZsq{}}\PY{p}{,} \PY{l+s+s1}{\PYZsq{}}\PY{l+s+s1}{?}\PY{l+s+s1}{\PYZsq{}}\PY{p}{,} \PY{l+s+s1}{\PYZsq{}}\PY{l+s+s1}{!}\PY{l+s+s1}{\PYZsq{}}\PY{p}{,} \PY{l+s+s1}{\PYZsq{}}\PY{l+s+s1}{*}\PY{l+s+s1}{\PYZsq{}}\PY{p}{\PYZcb{}}
        
        \PY{n}{msgs} \PY{o}{=} \PY{p}{(}\PY{l+s+s1}{\PYZsq{}}\PY{l+s+s1}{Length of the password is not satisfied}\PY{l+s+s1}{\PYZsq{}}\PY{p}{,}
         \PY{l+s+s1}{\PYZsq{}}\PY{l+s+s1}{At least one digit}\PY{l+s+s1}{\PYZsq{}}\PY{p}{,}
         \PY{l+s+s1}{\PYZsq{}}\PY{l+s+s1}{At least one symbole from * ! @ ?}\PY{l+s+s1}{\PYZsq{}}\PY{p}{,}
         \PY{l+s+s1}{\PYZsq{}}\PY{l+s+s1}{At least one uppercase letter}\PY{l+s+s1}{\PYZsq{}}\PY{p}{,}
         \PY{l+s+s1}{\PYZsq{}}\PY{l+s+s1}{At least one lowercase letter}\PY{l+s+s1}{\PYZsq{}}\PY{p}{)}
        
        \PY{n}{cond} \PY{o}{=} \PY{k+kc}{False}
        \PY{k}{while}\PY{p}{(}\PY{n}{cond} \PY{o}{==} \PY{k+kc}{False}\PY{p}{)}\PY{p}{:}
          \PY{n}{psw} \PY{o}{=} \PY{n+nb}{input}\PY{p}{(}\PY{l+s+s1}{\PYZsq{}}\PY{l+s+s1}{Define a password:}\PY{l+s+s1}{\PYZsq{}}\PY{p}{)}
          \PY{n}{psw\PYZus{}set} \PY{o}{=} \PY{n+nb}{set}\PY{p}{(}\PY{n}{psw}\PY{p}{)}
         \PY{c+c1}{\PYZsh{} a list to collect the results of constraints}
          \PY{n}{conds} \PY{o}{=} \PY{p}{[}\PY{l+m+mi}{0}\PY{p}{,}\PY{l+m+mi}{0}\PY{p}{,}\PY{l+m+mi}{0}\PY{p}{,}\PY{l+m+mi}{0}\PY{p}{,}\PY{l+m+mi}{0}\PY{p}{]}  
          
          \PY{n}{conds}\PY{p}{[}\PY{l+m+mi}{0}\PY{p}{]} \PY{o}{=} \PY{n+nb}{int}\PY{p}{(}\PY{n+nb}{len}\PY{p}{(}\PY{n}{psw}\PY{p}{)} \PY{o}{\PYZgt{}}\PY{o}{=} \PY{l+m+mi}{8} \PY{p}{)}
          \PY{n}{conds}\PY{p}{[}\PY{l+m+mi}{1}\PY{p}{]} \PY{o}{=} \PY{n+nb}{int}\PY{p}{(}\PY{n+nb}{len}\PY{p}{(}\PY{n}{psw\PYZus{}set}\PY{o}{.}\PY{n}{intersection}\PY{p}{(}\PY{n}{dgt}\PY{p}{)}\PY{p}{)} \PY{o}{\PYZgt{}} \PY{l+m+mi}{0}\PY{p}{)}
          \PY{n}{conds}\PY{p}{[}\PY{l+m+mi}{2}\PY{p}{]} \PY{o}{=} \PY{n+nb}{int}\PY{p}{(}\PY{n+nb}{len}\PY{p}{(}\PY{n}{psw\PYZus{}set}\PY{o}{.}\PY{n}{intersection}\PY{p}{(}\PY{n}{sym}\PY{p}{)}\PY{p}{)} \PY{o}{\PYZgt{}} \PY{l+m+mi}{0}\PY{p}{)}
          \PY{n}{conds}\PY{p}{[}\PY{l+m+mi}{3}\PY{p}{]} \PY{o}{=} \PY{n+nb}{int}\PY{p}{(}\PY{n+nb}{len}\PY{p}{(}\PY{n}{psw\PYZus{}set}\PY{o}{.}\PY{n}{intersection}\PY{p}{(}\PY{n}{upp}\PY{p}{)}\PY{p}{)} \PY{o}{\PYZgt{}} \PY{l+m+mi}{0}\PY{p}{)}
          \PY{n}{conds}\PY{p}{[}\PY{l+m+mi}{4}\PY{p}{]} \PY{o}{=} \PY{n+nb}{int}\PY{p}{(}\PY{n+nb}{len}\PY{p}{(}\PY{n}{psw\PYZus{}set}\PY{o}{.}\PY{n}{intersection}\PY{p}{(}\PY{n}{low}\PY{p}{)}\PY{p}{)} \PY{o}{\PYZgt{}} \PY{l+m+mi}{0}\PY{p}{)}
        
          \PY{c+c1}{\PYZsh{} A password is accepted if all the constraints are met.}
          \PY{n}{cond} \PY{o}{=} \PY{p}{(}\PY{n+nb}{sum}\PY{p}{(}\PY{n}{conds}\PY{p}{)}\PY{o}{==}\PY{n+nb}{len}\PY{p}{(}\PY{n}{conds}\PY{p}{)}\PY{p}{)}  
        
         \PY{c+c1}{\PYZsh{} if a condition is not satisfied, coresponding message is printed.}
          \PY{k}{if} \PY{o+ow}{not}\PY{p}{(}\PY{n}{cond}\PY{p}{)}\PY{p}{:} 
            \PY{n+nb}{print}\PY{p}{(}\PY{l+s+s1}{\PYZsq{}}\PY{l+s+s1}{invalid password: enter again}\PY{l+s+s1}{\PYZsq{}}\PY{p}{)}
            \PY{k}{for} \PY{n}{i} \PY{o+ow}{in} \PY{n+nb}{range}\PY{p}{(}\PY{l+m+mi}{0}\PY{p}{,}\PY{n+nb}{len}\PY{p}{(}\PY{n}{conds}\PY{p}{)}\PY{p}{)}\PY{p}{:}
                \PY{k}{if} \PY{n}{conds}\PY{p}{[}\PY{n}{i}\PY{p}{]}\PY{o}{==}\PY{l+m+mi}{0}\PY{p}{:}
                    \PY{n+nb}{print}\PY{p}{(}\PY{n}{msgs}\PY{p}{[}\PY{n}{i}\PY{p}{]}\PY{p}{)}
        
        \PY{n+nb}{print}\PY{p}{(}\PY{l+s+s1}{\PYZsq{}}\PY{l+s+s1}{Password is Valid.}\PY{l+s+s1}{\PYZsq{}}\PY{p}{)}
\end{Verbatim}


    \begin{Verbatim}[commandchars=\\\{\}]
Define a password:Ax23bvNqQ
invalid password: enter again
At least one symbole from * ! @ ?
Define a password:A23
invalid password: enter again
Length of the password is not satisfied
At least one symbole from * ! @ ?
At least one lowercase letter
Define a password:Ax23bvNqQ@0?\#
Password is Valid.

    \end{Verbatim}

    Wait a minute. Did you see what has happened in our previous test? Our
program has accepted Ax23bvNqQ@0?\# as a valid password. But, this is
not correct, right? The password contains \# which is not an element of
our accepted symbols. We are missing one more condition. Let's revise
our solution. Simply, our password set mut not have any other member
except \[ upp \cup low \cup sym \cup dgt \] This means if we remove all
the valid members from our password, nothing must remain:
\[ psw - (upp \cup low \cup sym \cup dgt)  = \{\} \]

We update our program here.

    \begin{Verbatim}[commandchars=\\\{\}]
{\color{incolor}In [{\color{incolor}6}]:} \PY{k+kn}{import} \PY{n+nn}{string}
        
        \PY{n}{upp} \PY{o}{=} \PY{n+nb}{set}\PY{p}{(}\PY{n}{string}\PY{o}{.}\PY{n}{ascii\PYZus{}uppercase}\PY{p}{)}
        \PY{n}{low} \PY{o}{=} \PY{n+nb}{set}\PY{p}{(}\PY{n}{string}\PY{o}{.}\PY{n}{ascii\PYZus{}lowercase}\PY{p}{)}
        \PY{n}{dgt} \PY{o}{=} \PY{n+nb}{set}\PY{p}{(}\PY{n}{string}\PY{o}{.}\PY{n}{digits}\PY{p}{)}
        \PY{n}{sym} \PY{o}{=} \PY{p}{\PYZob{}}\PY{l+s+s1}{\PYZsq{}}\PY{l+s+s1}{@}\PY{l+s+s1}{\PYZsq{}}\PY{p}{,} \PY{l+s+s1}{\PYZsq{}}\PY{l+s+s1}{?}\PY{l+s+s1}{\PYZsq{}}\PY{p}{,} \PY{l+s+s1}{\PYZsq{}}\PY{l+s+s1}{!}\PY{l+s+s1}{\PYZsq{}}\PY{p}{,} \PY{l+s+s1}{\PYZsq{}}\PY{l+s+s1}{*}\PY{l+s+s1}{\PYZsq{}}\PY{p}{\PYZcb{}}
        \PY{n}{max\PYZus{}len} \PY{o}{=} \PY{l+m+mi}{8}
        
        \PY{c+c1}{\PYZsh{} What is the benefit of defining this collection of messages as tuple?}
        \PY{n}{msgs} \PY{o}{=} \PY{p}{(}\PY{l+s+s1}{\PYZsq{}}\PY{l+s+s1}{Length of the password is not satisfied}\PY{l+s+s1}{\PYZsq{}}\PY{p}{,}
         \PY{l+s+s1}{\PYZsq{}}\PY{l+s+s1}{At least one digit}\PY{l+s+s1}{\PYZsq{}}\PY{p}{,}
         \PY{l+s+s1}{\PYZsq{}}\PY{l+s+s1}{At least one symbole from * ! @ ?}\PY{l+s+s1}{\PYZsq{}}\PY{p}{,}
         \PY{l+s+s1}{\PYZsq{}}\PY{l+s+s1}{At least one uppercase letter}\PY{l+s+s1}{\PYZsq{}}\PY{p}{,}
         \PY{l+s+s1}{\PYZsq{}}\PY{l+s+s1}{At least one lowercase letter}\PY{l+s+s1}{\PYZsq{}}\PY{p}{,}
         \PY{l+s+s1}{\PYZsq{}}\PY{l+s+s1}{Password contains and invalid symbol}\PY{l+s+s1}{\PYZsq{}}\PY{p}{)}
        
        \PY{n}{cond} \PY{o}{=} \PY{k+kc}{False}
        \PY{k}{while}\PY{p}{(}\PY{n}{cond} \PY{o}{==} \PY{k+kc}{False}\PY{p}{)}\PY{p}{:}
          \PY{n}{psw} \PY{o}{=} \PY{n+nb}{input}\PY{p}{(}\PY{l+s+s1}{\PYZsq{}}\PY{l+s+s1}{Define a password:}\PY{l+s+s1}{\PYZsq{}}\PY{p}{)}
          \PY{n}{psw\PYZus{}set} \PY{o}{=} \PY{n+nb}{set}\PY{p}{(}\PY{n}{psw}\PY{p}{)}
          \PY{n}{conds} \PY{o}{=} \PY{p}{[}\PY{l+m+mi}{0}\PY{p}{,}\PY{l+m+mi}{0}\PY{p}{,}\PY{l+m+mi}{0}\PY{p}{,}\PY{l+m+mi}{0}\PY{p}{,}\PY{l+m+mi}{0}\PY{p}{,}\PY{l+m+mi}{0}\PY{p}{]}  \PY{c+c1}{\PYZsh{} a list to collect the results of constraints}
          
          \PY{n}{conds}\PY{p}{[}\PY{l+m+mi}{0}\PY{p}{]} \PY{o}{=} \PY{n+nb}{int}\PY{p}{(}\PY{n+nb}{len}\PY{p}{(}\PY{n}{psw}\PY{p}{)} \PY{o}{\PYZgt{}}\PY{o}{=} \PY{n}{max\PYZus{}len} \PY{p}{)}
          \PY{n}{conds}\PY{p}{[}\PY{l+m+mi}{1}\PY{p}{]} \PY{o}{=} \PY{n+nb}{int}\PY{p}{(}\PY{n+nb}{len}\PY{p}{(}\PY{n}{psw\PYZus{}set}\PY{o}{.}\PY{n}{intersection}\PY{p}{(}\PY{n}{dgt}\PY{p}{)}\PY{p}{)} \PY{o}{\PYZgt{}} \PY{l+m+mi}{0}\PY{p}{)}
          \PY{n}{conds}\PY{p}{[}\PY{l+m+mi}{2}\PY{p}{]} \PY{o}{=} \PY{n+nb}{int}\PY{p}{(}\PY{n+nb}{len}\PY{p}{(}\PY{n}{psw\PYZus{}set}\PY{o}{.}\PY{n}{intersection}\PY{p}{(}\PY{n}{sym}\PY{p}{)}\PY{p}{)} \PY{o}{\PYZgt{}} \PY{l+m+mi}{0}\PY{p}{)}
          \PY{n}{conds}\PY{p}{[}\PY{l+m+mi}{3}\PY{p}{]} \PY{o}{=} \PY{n+nb}{int}\PY{p}{(}\PY{n+nb}{len}\PY{p}{(}\PY{n}{psw\PYZus{}set}\PY{o}{.}\PY{n}{intersection}\PY{p}{(}\PY{n}{upp}\PY{p}{)}\PY{p}{)} \PY{o}{\PYZgt{}} \PY{l+m+mi}{0}\PY{p}{)}
          \PY{n}{conds}\PY{p}{[}\PY{l+m+mi}{4}\PY{p}{]} \PY{o}{=} \PY{n+nb}{int}\PY{p}{(}\PY{n+nb}{len}\PY{p}{(}\PY{n}{psw\PYZus{}set}\PY{o}{.}\PY{n}{intersection}\PY{p}{(}\PY{n}{low}\PY{p}{)}\PY{p}{)} \PY{o}{\PYZgt{}} \PY{l+m+mi}{0}\PY{p}{)}
          \PY{c+c1}{\PYZsh{} our new condition is added}
          \PY{n}{conds}\PY{p}{[}\PY{l+m+mi}{5}\PY{p}{]} \PY{o}{=} \PY{n+nb}{int}\PY{p}{(}\PY{n+nb}{len}\PY{p}{(}\PY{n}{psw\PYZus{}set} \PY{o}{\PYZhy{}} \PY{p}{(}\PY{n}{low} \PY{o}{|} \PY{n}{upp} \PY{o}{|} \PY{n}{sym} \PY{o}{|} \PY{n}{dgt}\PY{p}{)}\PY{p}{)} \PY{o}{==} \PY{l+m+mi}{0}\PY{p}{)}  
        
          \PY{n}{cond} \PY{o}{=} \PY{p}{(}\PY{n+nb}{sum}\PY{p}{(}\PY{n}{conds}\PY{p}{)}\PY{o}{==}\PY{n+nb}{len}\PY{p}{(}\PY{n}{conds}\PY{p}{)}\PY{p}{)}  
        
          \PY{k}{if} \PY{o+ow}{not}\PY{p}{(}\PY{n}{cond}\PY{p}{)}\PY{p}{:} 
            \PY{n+nb}{print}\PY{p}{(}\PY{l+s+s1}{\PYZsq{}}\PY{l+s+s1}{invalid password: enter again}\PY{l+s+s1}{\PYZsq{}}\PY{p}{)}
            \PY{k}{for} \PY{n}{i} \PY{o+ow}{in} \PY{n+nb}{range}\PY{p}{(}\PY{l+m+mi}{0}\PY{p}{,}\PY{n+nb}{len}\PY{p}{(}\PY{n}{conds}\PY{p}{)}\PY{p}{)}\PY{p}{:}
                \PY{k}{if} \PY{n}{conds}\PY{p}{[}\PY{n}{i}\PY{p}{]}\PY{o}{==}\PY{l+m+mi}{0}\PY{p}{:}
                    \PY{n+nb}{print}\PY{p}{(}\PY{n}{msgs}\PY{p}{[}\PY{n}{i}\PY{p}{]}\PY{p}{)}
        
        \PY{n+nb}{print}\PY{p}{(}\PY{l+s+s1}{\PYZsq{}}\PY{l+s+s1}{Password is Valid.}\PY{l+s+s1}{\PYZsq{}}\PY{p}{)}
\end{Verbatim}


    \begin{Verbatim}[commandchars=\\\{\}]
Define a password:Ax23bvNqQ@0?\#
invalid password: enter again
Password contains and invalid symbol
Define a password:Ax23bvNqQ@0?
Password is Valid.

    \end{Verbatim}

    \textbf{Conclusion}: In this simple example we tried to present how
step-by-step we can tackle the problem, applying mathematical concepts
sketch the solution, define the core of our algorithm, implement the
code and validate the results. This skill in problem solving is one of
the fundamental skills for a student in computer science.


    % Add a bibliography block to the postdoc
    
    
    
    \end{document}
